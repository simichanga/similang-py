\chapter{Indicații de redactare}
\label{chap:intro}

\myLettrine{L}{ucrarea} de diplomă este subiectul examenului de finalizare a studiilor și trebuie să reflecte contribuţiile personale ale autorului, i.e. studentul candidat. Lucrarea de diplomă descrie un proiect realizat de autor sub îndrumarea cadrului didactic ales, de obicei, la începutul anului terminal al studiilor de licență. Examenul de finalizare al studiilor se desfășoară conform procedurilor indicate la aviziere și prin website-ul facultăţii. Ghidul de față este valabil pentru studenții direcției B: Ingineria Sistemelor.

\textbf{Respectarea} acestui ghid este \textbf{obligatorie}.

\section{Conținutul lucrării de diplomă}

Lucrarea de diplomă trebuie să reflecte capacitatea autorului de a înțelege și utiliza conceptele din subdomeniul specific al proiectului ales, capacitatea acestuia de a sintetiza cunoștințele acumulate pe parcursul studiilor, și de a le prezenta într-o manieră riguroasă. Lucrarea va fi scrisă într-una din limbile română sau engleză, la alegerea autorului, după consultarea profesorului coordonator.

\textbf{Lucrarea de diplomă se redactează la persoana I singular, pentru a evidenția activitățile autorului} vs. contribuţiile altor cercetători în domeniu ce stau la baza proiectului.

Spre exemplu, ``\emph{Pentru proiectarea algoritmului de reglare, am ales inițial o metodă în frecvență}.'' se referă la contribuția proprie a autorului, în timp ce ``\emph{Pentru proiectarea algoritmului de reglare s-a ales inițial o metodă în frecvență (Popescu, 2000)}.'' reflectă contribuția lucrării citate prin (Popescu, 2000) la subiect.


Lucrarea de diplomă are între 30 şi 40 de pagini. În aceste pagini nu se numără pagina de titlu, cuprinsul, anexele, bibliografia și lista contribuţiilor personale.

Lucrarea de diplomă este compusă din:

\begin{enumerate}[label=\alph*.]
	\item Pagină de titlu
	\item Cuprins
	\item Introducere
	\item Capitole - corpul lucrării
	\item Concluzii
	\item Anexe (dacă este cazul)
	\item Bibliografie  
	\item Lista contribuțiilor personale
\end{enumerate}
La alegere, se pot include liste ale figurilor, tabelelor și algoritmilor.

\section{Introducere și concluzii}

Capitolul introductiv al lucrării de diplomă are o lungime de 1-2 pagini, cuprinzând scopul și obiectivele lucrării. La alegere, se poate include și motivația alegerii temei. Se recomandă descrierea cât mai clară a obiectivelor lucrării.

Concluziile lucrării rezumă îndeplinirea obiectivelor lucrării prin prisma activității desfășurate pe parcusul realizării proiectului. Acestea conțin opiniile proprii ale autorului, susținute de rezultatele prezentate în corpul lucrării. Capitolul concluziilor nu este, de obicei, mai lung de 1 pagină.

\section{Corpul lucrării}

Structura capitolelor asociate corpului lucrării este la alegerea autorului, cu ilustrarea următoarelor componente: formularea problemei, soluția abordată, implementare (software și/sau hardware), rezultate / exemplu numeric.

Corpul lucrării de diplomă trebuie să conțină în principal contribuţiile autorului, urmând pașii logici de proiectare conform temei alese.

\textbf{Elementele așa-zise ``teoretice'' ce reflectă cunoștințe din domeniu deja cunoscute trebuie să nu depășească 1/3 din corpul lucrării. Se recomandă includerea exclusivă a acelor componente absolut necesare pentru prezentarea contribuțiilor proprii.} Spre exemplu, în cazul proiectării unui sistem de conducere, este necesară menționarea indicatorilor de performanță ce vor fi utilizați pentru validarea sistemului; în schimb, nu este necesară includerea testului de stabilitate Hurwitz, ci numai aplicarea acestuia în contextul studiat.

\textbf{Schemele și structurile specifice proiectului (fie ele din domeniul reglării, al proiectării hardware, sau al sistemelor software) trebuie sa se limiteze numai la cele ce țin strict de tema și subiectul lucrării.} Spre exemplu, în cazul proiectării unei aplicații pentru reglarea prin comandă wireless a unui proces, este relevantă schema structurii programului proiectat, însă nu este necesară descrierea în detaliu a protocolului de comunicație prin bluetooth. 

În cazurile în care tema lucrării de diplomă, la indicațiile îndrumătorului, continuă/completează/extinde rezultatele unei lucrări științifice existente, iar acestea sunt relativ noi sau nu se regăsesc în curicula uzuală a anilor de studiu, este permisă rezumarea, pe scurt, a acestora în scopul creșterii lizibilității lucrării.

\textbf{Este interzisă copierea formulelor, ecuațiilor, figurilor, imaginilor etc. din surse externe, inclusiv cursuri.} Toate aceste elemente incluse în lucrarea de diplomă trebuie să fie redactate / desenate de autor. Este absolut obligatorie citarea surselor acestor elemente. În cazul figurilor, se admite o singură excepție per lucrare, numai atunci când imaginea respectivă descrie un rezultat al altor cercetători (publicat întro
lucrare științifică citată).

\section{Anexe}

Anexele lucrării de diplomă conțin informație-suport care susține realizarea proiectului, cum ar fi: teoreme, formule de calcul uzuale, fișe tehnice pentru componente (senzori, plăci etc.), extrase de cod, scheme electrice, componente intermediare relevante din parcusul realizării proiectului, descriere IDE-uri, limbaje de programare, algoritmi clasici.

\section{Lista contribuţiilor personale}

Lista contribuţiilor personale este \textbf{obligatorie}. Aceasta este un tabel redactat pe 1 pagină inserată la finalul lucrării. Lista enumeră contribuțiile personale (de ex. documentare, implementare cod și/sau soluție hardware, testare, etc), probleme întâmpinate în proiectare, precum și durata de realizare a fiecărui element enumerate anterior. Acest tabel este echivalent cu borderoul caietului de laborator sau a jurnalului de activități (engl. \emph{log}) utilizat pe parcursul realizării proiectului.

Durata este exprimată în zile echivalente de lucru, adică 1 zi = 8 ore.

Tabelul \ref{tab:lista_contrib} arată forma recomandată de prezentare a contribuțiilor personale.

\begin{table}[ht!]
\begin{tabular}{|c|c|c|}
\hline
\multicolumn{3}{|c|}{\textbf{Titlu lucrare de diplomă}} \\
\multicolumn{3}{|c|}{\textbf{Nume autor}} \\
\multicolumn{3}{|c|}{\textbf{Nume coordonator/îndrumător}} \\
\hline
 & Activitate & Durată [zile] \\
\hline
1 & \hspace*{12cm} & \\
\hline
2 &  & \\
\hline
3 &  & \\
\hline
\end{tabular}
\centering
\caption{Lista contribuțiilor personale}
\label{tab:lista_contrib}
\end{table}

\section{Redactarea lucrării de diplomă}

\textbf{Lucrarea de diplomă nu este un manuscris, ci un produs finit, prezentarea acestuia necesitând un anumit grad de finisare în formatare.}

Formatul uzual al paginilor pentru redactarea lucrării de diplomă are următoarele caracteristici:

\begin{itemize}
\item pagină A4
\item 	margini de 2cm sus, jos \c si la dreapta
\item  margine de 3cm la stânga
\item spaţiere simplă, la un rând (\emph{single line})
\end{itemize}

\textbf{Acest document respectă formatul propus.}

Fonturile cele mai lizibile pentru redactarea lucrării de diplomă sunt acele
fonturi care au corpul literei de dimensiune echilibrată în lăţime şi înălţime. Exemple de fonturi care se pretează redactării lucrării de diplomă sunt: Times New Roman 12pt, Arial 12pt, Verdana 11pt, Adobe Caslon Pro12 pt, Linotype Palatino 12pt, Helvetica 12pt, Neutra Text 12pt, Kozuka Mincho 11pt.

Nu se vor utiliza fonturi de dimensiune mai mare decât 12pt în corpul lucrării, excepție făcând titlurile capitolelor. Se recomandă alegerea unui font care conţine diacritice, în cazul redactării lucrării în limba română.

Lucrarea de diplomă se redactează, în întregime, cu acelaşi font. Excepţie fac anexele, unde este posibilă utilizarea unui font special pentru transcrierea scripturilor şi a programelor, de exemplu: Courier şi/sau Courier New cu dimensiune de 10 sau 11pt.

Paragrafele se despart printr-un rând liber. Începutul unui paragraf se marchează prin deplasarea la dreapta a primului rând din paragraf, de obicei cu 1 sau 1.5 cm. Corpurile de text se distribuie pe orizontală de la un capăt al celuilalt al paginii (aliniere \emph{justified}), și nu la stânga. 

\subsection{Pagina de titlu} 
Pagina de titlu conţine numele lucrării de diplomă, numele autorului și al coordonatorului acestuia, numele universităţii/facultăţii/departamentului, orașul și anul în care a fost scrisă aceasta. Prima pagina a acestui document prezintă o sugestie de formatare a paginii de titlu pentru lucrările de diplomă.

\subsection{Cuprinsul lucrării de diplomă} 
Cuprinsul lucrării de diplomă conţine toate titlurile capitolelor, secţiunilor şi subsecţiunilor, în ordinea în care acestea apar în lucrare. Se recomandă să nu se prescurteze cuvintele "Capitol" şi "Secțiune" în cazul în care acestea sunt utilizate înainte de numărul capitolului şi al secţiunii sau subsecţiunii respective. Uzual, aceste cuvinte se omit.

\subsection{Numerotare}

Paginile lucrării se numerotează în ordine. Nu este indicată reînceperea numerotării paginilor cu fiecare capitol. De asemenea, nu este indicată numerotarea paginii de titlu.

Numerele de pagini se includ în câmpuri speciale de subsol (Footer), în care fontul utilizat trebuie să fie același cu restul lucrării și cu 1 sau 2 puncte tipografice mai mic. Optional, se poate include un câmp conţinând titlul lucrării în zona superioară a paginii (Header), acesta necesitând aceeași dimensiune de font adoptată pentru numerele de pagini.

\subsection{Figuri, grafice și tabele}

 Figurile și tabelele trebuie să aibă un titlu care să menţioneze tipul obiectului respectiv, conţinutul acestuia și numărul acestuia în cadrul capitolului:

\begin{description}
\item[Figura c.n.] desemnează o figură, c fiind identificatorul capitolului, iar n reprezentând numărul figurii în cadrul acelui capitol; acest titlu va fi urmat de numele figurii, descriind conţinutul acesteia. De exemplu: Figura 3.2. Sistem de reglare automată a presiunii va fi titlul figurii a doua din capitolul 3, conţinând structura unui sistem de reglare automată a presiunii. 
\item[Tabelul c.n.] desemnează un tabel, c fiind identificatorul capitolului, iar n reprezentând numărul tabelului în cadrul acelui capitol; acest titlu va fi urmat de numele tabelului, descriind conţinutul acestuia. De exemplu: Tabelul 5.6. Caracteristici tehnice ale traductorului de temperatură va fi titlul tabelului al șaselea din capitolul 5, conţinând caracteristicile tehnice ale unui traductor de temperatură.
\end{description}
Graficele sunt considerate figuri și vor purta titluri adecvate. Graficele trebuie să aibă o etichetă pe fiecare axă, descriind semnificaţia acesteia, menţionând unitatea de măsură acolo unde este cazul. De exemplu, pentru răspunsul în timp al unui sistem de ordinul I oarecare, este îndeajuns a atașa eticheta y pe ordonată și eticheta t pe abscisă. Însă dacă acest răspuns aparţine unui model al unui proces fizic, se va menţiona unitatea de măsură pe fiecare axă, de exemplu y[m] și t[s].

Pentru o tipărire corectă, toate figurile și graficele ar trebui salvate la o rezoluţie de cel putin 300dpi pentru cele color și 100dpi pentru cele alb-negru. Se recomandă salvarea acestora în format .tiff sau .png pentru conservarea calităţii imaginilor.

Se recomandă alinierea centrală a figurilor. Tabelele se pot alinia la stânga, lăsând faţă de marginea paginii (acolo unde este posibil si dacă tabelul nu acoperă toată lăţimea paginii) aceeași dimensiune ca și în cazul primului rând al paragrafelor.

\subsection{Ecuaţii}

	Ecuaţiile se scriu cu aceeași înălţime de font ca și corpul textului și se numerotează în ordinea apariţiei în text: (c.n) unde c reprezintă identificatorul capitolului curent, iar n este numărul ecuaţiei în capitol. Ecuaţiile pot avea eticheta de identificare la stânga sau la dreapta. Ecuaţiile se pot alinia centrat sau la stânga. De exemplu:

\be
\label{eq:test}
5+x=0
\ee
unde 1 reprezintă numărul capitolului, iar 1 este numărul ecuaţiei în cadrul acestuia. Înainte și după fiecare ecuaţie se lasă un rând liber.


\subsection{Bibliografie}

	Lista bibliografică este o componentă esenţială a lucrării de diplomă, aceasta demonstrând documentarea efectuată de către autor și marcând corespunzător ideile care nu îi aparţin acestuia. Bibliografia este formată dintr-o listă ordonată alfabetic. \textbf{Toate} elementele acestei liste trebuie \textbf{citate în text}.


